\chapter{Imutabilidade e mudança}
\label{les:1}

\begin{chapquote}{Alice}
\enquote{Eu me pergunto se fui trocada de noite. Deixe-me pensar. Era eu mesma quando eu
acordei esta manhã? Eu quase acho que me lembro de me sentir um pouco diferente.
Mas se eu não sou a mesma, a próxima pergunta é 'Quem sou eu?' Ah,
esse é o grande quebra-cabeça!}
\end{chapquote}

Bitcoin é inerentemente difícil de descrever. Ele é \textit{uma coisa}, e qualquer
tentativa de fazer uma comparação com conceitos anteriores - seja chamando
o ouro digital ou a internet do dinheiro - provavelmente ficará aquém do todo. Qualquer que seja sua analogia favorita, dois aspectos do
Bitcoin é absolutamente essencial: descentralização e imutabilidade.

\paragraph{}
Uma maneira de pensar sobre o Bitcoin é como um contrato social automatizado\footnote{Hasu,
Unpacking Bitcoin's Social Contract~\cite{social-contract}}. O software é
apenas uma peça do quebra-cabeça, e esperando mudar o Bitcoin mudando o
software é um exercício de futilidade. Alguém teria que convencer o resto do
rede para adotar as mudanças, o que é mais um esforço psicológico do que um
engenharia de software.

\paragraph{}
A seguir, pode parecer absurdo a princípio, como muitas outras coisas neste espaço, mas acredito que é profundamente verdadeiro:
você não mudará o Bitcoin, mas o Bitcoin mudará você.

\begin{quotation}\begin{samepage}
\enquote{Bitcoin irá mudar-nos mais do que nós o mudaremos.}
\begin{flushright} -- Marty Bent\footnote{Tales From the Crypt~\cite{tftc21}}
\end{flushright}\end{samepage}\end{quotation}

Levei muito tempo para perceber a profundidade disso. Sendo o Bitcoin
apenas um software e o código fonte aberto, você pode simplesmente mudar
coisas à vontade, certo?
Errado. \textit{Muito} errado. Sem surpresa o criador do Bitcoin sabia muito bem disso.

\begin{quotation}\begin{samepage}
\enquote{A natureza do Bitcoin é tal que, uma vez que a versão 0.1 foi lançada, o desenho central será escrito em pedra para o resto de sua vida.}
\begin{flushright} -- Satoshi Nakamoto\footnote{BitcoinTalk forum post: `Re:
Transactions and Scripts\ldots'~\cite{satoshi-set-in-stone}}
\end{flushright}\end{samepage}\end{quotation}

Muitas pessoas tentaram mudar a natureza do Bitcoin. Até agora, todos
elas falharam. Embora haja um mar interminável de forks (ramificações) e altcoins,
a rede Bitcoin continua funcionando naturalmente, assim como fez quando o primeiro
nó ficou online. As altcoins não importarão a longo prazo. Os forks (ramificações)
acabarão morrendo de fome. Bitcoin é o que importa. Enquanto nosso entendimento fundamental de matemática e / ou física não muda,
o Honeybadger Bitcoin continuará a não se importará.


\begin{quotation}\begin{samepage}
\enquote{Bitcoin é o primeiro exemplo de uma nova forma de vida. Vive e
respira na internet. Ele vive porque pode pagar as pessoas para o manter
vivo. [\ldots] Não pode ser mudado. Isso não está aberto a discussão. isto
não pode ser adulterado. Não pode ser corrompido. Não pode ser parado.
[\ldots] Se a guerra nuclear destruísse metade do nosso planeta, ele continuaria
a viver, sem corrupção.}
\begin{flushright} -- Ralph Merkle\footnote{DAOs, Democracy and
Governance,~\cite{merkle-dao}}
\end{flushright}\end{samepage}\end{quotation}

Os batimentos cardíacos da rede Bitcoin durarão mais que todos os nossos.
~

Perceber o que foi dito acima me mudou muito mais do que os blocos anteriores do blockchain Bitcoin. Isso mudou minha preferência de tempo, minha compreensão de
economia, minhas opiniões políticas e muito mais. Inferno, está até mudando a
dietas das pessoas\footnote{Inside the World of the Bitcoin
Carnivores,~\cite{carnivores}}. Se tudo isso parece loucura para você, você está
em boa companhia. Tudo isso é loucura, e ainda está acontecendo.

~

\paragraph{O Bitcoin me ensinou que não vai mudar. Eu vou.}

% ---
%
% #### Through the Looking-Glass
%
% - [Bitcoin's Gravity: How idea-value feedback loops are pulling people in][gravity]
% - [Lesson 18: Move slowly and don't break things][lesson18]
%
% #### Down the Rabbit Hole
%
% - [Unpacking Bitcoin's Social Contract][automated social contract]: A framework for skeptics by Hasu
% - [DAOs, Democracy and Governance][Ralph Merkle] by Ralph C. Merkle
% - [Marty's Bent][bent]: A daily newsletter highlighting signal in Bitcoin by Marty Bent
% - [Technical Discussion on Bitcoin's Transactions and Scripts][Satoshi Nakamoto] by Satoshi Nakamoto, Gavin Andresen, and others
% - [Inside the World of the Bitcoin Carnivores][carnivores]: Why a small community of Bitcoin users is eating meat exclusively by Jordan Pearson
% - [Tales From the Crypt][tftc] hosted by Marty Bent
%
% <!-- Internal -->
% [gravity]: 
% [lesson18]: {{ 'bitcoin/lessons/ch3-18-move-slowly-and-dont-break-things' | absolute_url }}
%
% <!-- Further Reading -->
% [automated social contract]: https://medium.com/@hasufly/bitcoins-social-contract-1f8b05ee24a9
% [carnivores]: https://motherboard.vice.com/en_us/article/ne74nw/inside-the-world-of-the-bitcoin-carnivores
% [tftc]: https://tftc.io/tales-from-the-crypt/
% [bent]: https://tftc.io/martys-bent/
%
% <!-- Quotes -->
% [Ralph Merkle]: http://merkle.com/papers/DAOdemocracyDraft.pdf
% [Satoshi Nakamoto]: https://bitcointalk.org/index.php?topic=195.msg1611#msg1611
%
% <!-- Twitter People -->
% [Marty Bent]: https://twitter.com/martybent
%
% <!-- Wikipedia -->
% [alice]: https://en.wikipedia.org/wiki/Alice%27s_Adventures_in_Wonderland
% [carroll]: https://en.wikipedia.org/wiki/Lewis_Carroll
