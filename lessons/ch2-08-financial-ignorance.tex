\chapter{Ignorância Financeira}
\label{les:8}

\begin{chapquote}{Lewis Carroll, \textit{Alice no País das Maravilhas}}
\enquote{E que menininha ignorante ela vai achar que eu sou! 
Não, é melhor não perguntar nada: talvez eu veja o nome escrito em algum lugar.}
\end{chapquote}

Uma das coisas mais surpreendentes, para mim, foi a quantidade de financiamento,
economia e psicologia necessárias para entender o que a princípio
olhar parece ser puramente \textit{técnico} sistema --- um rede de computador.
Parafraseando um rapaz com pés peludos: \enquote{É um negócio perigoso, Frodo, entrar no Bitcoin. 
Você lê o whitepaper e, se não mantiver os pés no chão, não há como saber para onde pode ser levado.}

Para entender um novo sistema monetário, você precisa se familiarizar com o antigo. 
Comecei a perceber muito rápido que a quantidade de educação financeira que eu tive no
sistema educacional era essencialmente \textit{zero}.

\paragraph{}
Como uma criança de cinco anos, comecei a me fazer muitas perguntas: como o
sistema bancário funciona? Como funciona o mercado de ações? O que é dinheiro fiduciário? 
O que é o dinheiro \textit{normal} ? Porque existe tanta dívida ?
\footnote{\url{https://www.usdebtclock.org/}}Quanto dinheiro é realmente
impresso, e quem decide isso?
\newpage

Depois de um pânico leve sobre o escopo da minha ignorância, descobri
que eu estava em boa companhia.

\begin{quotation}\begin{samepage}
\enquote{Não é irônico que o Bitcoin tenha me ensinado mais sobre dinheiro do que todos 
os anos que passei trabalhando para instituições financeiras? \ldots incluindo começar 
minha carreira no banco central}
\begin{flushright} -- Aaron\footnote{Aaron (\texttt{@aarontaycc}, \texttt{@fiatminimalist}), tweet from Dec.
12, 2018~\cite{aarontaycc-tweet}}
\end{flushright}\end{samepage}\end{quotation}

\begin{quotation}\begin{samepage}
\enquote{Aprendi mais sobre finanças, economia, tecnologia, criptografia, recursos humanos
psicologia, política, teoria dos jogos, legislação e eu nos últimos três
meses de criptografia que nos últimos três anos e meio de faculdade}
\begin{flushright} -- Dunny\footnote{Dunny (\texttt{@BitcoinDunny}), tweet from Nov. 28,
2017~\cite{bitcoindunny-tweet}}
\end{flushright}\end{samepage}\end{quotation}

Estas são apenas duas das muitas confissões em todo o twitter.\footnote{Veja
\url{http://bit.ly/btc-learned} para mais confissões no twitter.} Bitcoin, como
explorado na Lição \ref{les:1}, é algo vivo. Mises argumentou que
a economia também é uma coisa viva. E como todos sabemos por experiência pessoal,
os seres vivos são inerentemente difíceis de entender.

\begin{quotation}\begin{samepage}
\enquote{Um sistema científico é apenas uma estação de uma busca infinita por conhecimento. 
É necessariamente afetado pela insuficiência inerente a todo esforço humano. Mas reconhecer esses fatos não
não significa que a economia atual esteja atrasada.
Significa apenas que a economia é uma coisa viva - e viver implica imperfeição e mudança.}
\begin{flushright} -- Ludwig von Mises\footnote{Ludwig von Mises, \textit{Human Action}
\cite{human-action}}
\end{flushright}\end{samepage}\end{quotation}

\newpage

We all read about various financial crises in the news, wonder about how
these big bailouts work and are puzzled over the fact that no one ever
seems to be held accountable for damages which are in the trillions. I
am still puzzled, but at least I am starting to get a glimpse of what is
going on in the world of finance.

Some people even go as far as to attribute the general ignorance on
these topics to systemic, willful ignorance. While history, physics,
biology, math, and languages are all part of our education, the world of
money and finance surprisingly is only explored superficially, if at
all. I wonder if people would still be willing to accrue as much debt as
they currently do if everyone would be educated in personal finance and
the workings of money and debt. Then I wonder how many layers of
aluminum make an effective tinfoil hat. Probably three.

Todos nós lemos sobre várias crises financeiras no noticiário, 
imaginamos como esses grandes resgates funcionam e ficamos intrigados 
com o fato de que ninguém parece ser responsabilizado por danos que estão na 
casa dos trilhões. Ainda estou confuso, mas pelo menos estou começando a ter 
uma idéia do que está acontecendo no mundo das finanças.

Algumas pessoas chegam a atribuir a ignorância geral sobre esses tópicos à 
ignorância sistêmica e voluntária. Embora história, física, biologia, matemática
e idiomas façam parte de nossa educação, o mundo do dinheiro e das finanças 
surpreendentemente só é explorado superficialmente, se é que existe. Gostaria
de saber se as pessoas ainda estariam dispostas a acumular tanta dívida como 
atualmente, se todos fossem educados em finanças pessoais e no funcionamento 
de dinheiro e dívida. Então me pergunto quantas  camadas de alumínio formam 
um chapéu de papel alumínio eficaz. Provavelmente três.

\begin{quotation}\begin{samepage}
\enquote{Essas quebras, esses resgates, não são acidentes. E nem é um
acidente que não há educação financeira na escola. [...] Está
premeditado. Assim como antes da Guerra Civil, era ilegal educar um escravo,
não temos permissão para aprender sobre dinheiro na escola.}
\begin{flushright} -- Robert Kiyosaki\footnote{Robert Kiyosaki, \textit{Why the Rich
are Getting Richer}\cite{robert-kiyosaki}}
\end{flushright}\end{samepage}\end{quotation}

Como em O Mágico de Oz, somos instruídos a não prestar atenção ao homem por trás da
cortina. Ao contrário de O Mágico de Oz, agora temos verdadeira
magia\footnote{\url{http://bit.ly/btc-wizardry}}: uma rede de transferência de 
valor aberta, sem fronteiras e resistente à censura. Não há cortina, e a mágica 
é visível para qualquer um.\footnote{\url{https://github.com/bitcoin/bitcoin}}

\paragraph{O Bitcoin me ensinou a olhar para trás da cortina e enfrentar minha ignorância financeira.}

% ---
%
% #### Down the Rabbit Hole
%
% - [Human Action][Ludwig von Mises] by Ludwig von Mises
% - [Why the Rich are Getting Richer][Robert Kiyosaki] by Robert Kiyosaki
%
% [real wizardry]: https://external-preview.redd.it/8d03MWWOf2HIyKrT8ThBGO4WFv-u25JaYqhbEO9b1Sk.jpg?width=683&auto=webp&s=dc5922d84717c6a94527bafc0189fd4ca02a24bb
% [visible to anyone]: https://github.com/bitcoin/bitcoin
%
% <!-- Wikipedia -->
% [alice]: https://en.wikipedia.org/wiki/Alice%27s_Adventures_in_Wonderland
% [carroll]: https://en.wikipedia.org/wiki/Lewis_Carroll
