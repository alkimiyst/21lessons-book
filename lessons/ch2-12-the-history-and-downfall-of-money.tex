\chapter{A História e a Queda do Dinheiro}
\label{les:12}

\begin{chapquote}{Lewis Carroll, \textit{Alice no País das Maravilhas}}
\enquote{Eles não se lembrariam das regras simples que os amigos lhes haviam dado, 
como, por exemplo, que, se você entrar no fogo, ele o queimará; e, se você cortar
o dedo profundamente com uma faca, geralmente ele sangra e ela nunca havia 
esquecido que, se você bebe uma garrafa marcada como 'veneno', é quase certo que 
discorde de você, mais cedo ou mais tarde.}
\end{chapquote}

Muitas pessoas pensam que o dinheiro é lastreado em ouro, trancado em grandes cofres, 
protegido por grossas paredes. Isso deixou de ser verdade há muitas décadas. Não sei 
ao certo o que pensei, pois estava com problemas muito mais profundos, praticamente 
sem entender ouro, dinheiro de papel ou por que ele precisaria ser apoiado por algo 
em primeiro lugar.

Uma parte de aprender sobre Bitcoin é aprender sobre dinheiro fiduciário: o que 
significa, como surgiu e por que pode não ser a melhor ideia que já tivemos. Então, 
o que exatamente é dinheiro fiduciário? E como acabamos usando?

Se algo é imposto por \textit{fiduciária}, significa simplesmente que é imposto por 
autorização ou proposição formal. Assim, dinheiro fiduciário é dinheiro simplesmente 
porque \textit{alguém} diz que é dinheiro. Como todos os governos usam hoje a moeda 
fiduciária, esse alguém é \textit{seu} governo. Infelizmente, você não é \textit{livre}
para discordar dessa proposta de valor. Você sentirá rapidamente que essa proposição 
é tudo, menos violenta. Se você se recusar a usar esta moeda em papel para fazer 
negócios e pagar impostos, as únicas pessoas com quem você poderá discutir economia 
serão seus colegas de cela.

O valor do dinheiro fiduciário não deriva de suas propriedades inerentes. A qualidade 
de um determinado tipo de moeda fiduciária é apenas correlacionada à (in) estabilidade
política e fiscal daqueles que a sonham existir. Seu valor é imposto por decreto, 
arbitrariamente.

\begin{figure}
  \centering
  \includegraphics[width=8cm]{assets/images/fiat-definition.png}
  \caption{fiat --- `Let it be done'}
  \label{fig:fiat-definition}
\end{figure}

\paragraph{}
Até recentemente, dois tipos de dinheiro eram usados: \textbf{dinheiro de commodities}, 
feito de precioso \textit{coisas} e \textbf{dinheiro representativo}, que simplesmente 
\textit{representa} o precioso, principalmente por escrito.

\paragraph{}
Já mencionamos o dinheiro das commodities acima. As pessoas usavam ossos especiais, 
conchas do mar e metais preciosos como dinheiro. Mais tarde, principalmente moedas 
feitas de metais preciosos como ouro e prata foram usadas como dinheiro. A moeda 
mais antiga encontrada até agora é feita de uma mistura natural de ouro e prata e 
foi feita há mais de 2700 anos. \footnote{De acordo com o historiador grego 
Heródoto, escrevendo no século V aC, os lídia foram os primeiros a usarem moedas 
de ouro e prata. \cite{coinage-origins}} Se algo é novo no Bitcoin, 
o conceito de moeda não é esse.

\newpage

\begin{figure}
  \centering
  \includegraphics[width=5cm]{assets/images/lydian-coin-stater.png}
  \caption{Lydian electrum coin. Picture cc-by-sa Classical Numismatic Group, Inc.}
  \label{fig:lydian-coin-stater}
\end{figure}

Acontece que acumular moedas, ou hodling, para usar a linguagem de hoje, é quase 
tão velho quanto moedas. O primeiro hodler de moedas foi alguém que colocou quase 
cem dessas moedas em uma panela e as enterrou nas fundações de um templo, apenas 
para ser encontrado 2500 anos depois. Muito bom armazenamento a frio, se você 
me perguntar.

Uma das desvantagens do uso de moedas de metal precioso é que elas podem ser cortadas, 
degradando efetivamente o valor da moeda. Novas moedas podem ser cunhadas a partir dos 
recortes, aumentando o suprimento de dinheiro ao longo do tempo, desvalorizando cada 
moeda individual no processo. As pessoas estavam literalmente tirando o máximo que 
pudessem conseguir com seus dólares de prata.
Gostaria de saber que tipo de \textit{Dollar Shave Club} anúncios eles tinham na época.

Uma vez que os governos recorrem a inflação descaradamente se eles são os únicos ao fazê-lo, 
foram envidados esforços para impedir esta degradação da guerrilha. Na moda clássica de 
policiais e ladrões, os cortadores de moedas ficam cada vez mais criativos com suas técnicas, 
forçando os \enquote{mestres da casa da moeda} a se tornarem ainda mais criativos com suas 
contramedidas. Isaac Newton, o físico de renome mundial da \textit {Principia Mathematica} 
fama, costumava ser um desses mestres. Ele é atribuído ao adicionar pequenas faixas ao lado 
das moedas que ainda hoje estão presentes. Foram-se os dias de barbear fácil.

\begin{figure}
  \includegraphics{assets/images/clipped-coins.png}
  \caption{Clipped silver coins of varying severity.}
  \label{fig:clipped-coins}
\end{figure}

Mesmo com esses métodos de degradação de moedas \footnote{Além de recortar, suar (sacudir as 
moedas em uma bolsa e recolher a poeira gasta) e entupir (perfurar um buraco no meio e martelar 
a moeda para fechar o buraco) eram os mais métodos proeminentes de degradação de moedas. 
\cite{wiki: coin-degradation}} mantido sob controle, as moedas ainda sofrem com outros problemas. 
Eles são volumosos e não muito conveniente de transportar, especialmente quando grandes 
transferências de valor precisam acontecer. Aparecer com uma enorme sacola de dólares de prata 
toda vez que você quiser comprar um Mercedes não é muito prático.

Falando em coisas alemãs: como os EUA \textit{dollar} receberam seu nome é outra história 
interessante. A palavra \enquote{dollar} é derivada da palavra alemã \textit{Thaler}, abreviação 
de \textit{Joachimsthaler}~\cite{wiki: thaler}. Um Joachimsthaler era uma moeda cunhada na 
cidade de \textit{Sankt Joachimsthal}. Thaler é simplesmente uma abreviação para alguém (ou algo)
vindo do vale, e como Joachimsthal era o vale da produção de moedas de prata, as pessoas 
simplesmente se referiam a essas moedas de prata como \textit{Thaler.} Thaler (alemão) se transformou 
em daalders (holandês) e finalmente dólares (em inglês).


\begin{figure}
  \centering
  \includegraphics[width=5cm]{assets/images/joachimsthaler.png}
  \caption{The original `dollar'. Saint Joachim is pictured with his robe and wizard hat. Picture cc-by-sa Wikipedia user Berlin-George}
  \label{fig:joachimsthaler}
\end{figure}

A introdução de dinheiro representativo anunciou a queda do dinheiro duro. Os certificados 
de ouro foram introduzidos em 1863 e, cerca de quinze anos depois, o dólar de prata também 
foi lenta mas seguramente sendo substituído por um proxy de papel: o certificado de prata. 
\cite{wiki: certificado de prata}

Demorou cerca de 50 anos desde a introdução dos primeiros certificados de prata até que 
esses pedaços de papel se transformassem em algo que hoje reconheceríamos como um dólar 
americano.

\begin{figure}
  \centering
  \includegraphics{assets/images/us-silver-dollar-note-smaller.png}
  \caption{A 1928 U.S. silver dollar. `Payable to the bearer on demand.' Picture cc-by-sa National Numismatic Collection at the Smithsonian Institution}
  \label{fig:us-silver-dollar-note-smaller}
\end{figure}

Observe que o dólar de prata de 1928 na Figura ~\ref{fig: us-nota-de-dólar-prateado-menor} 
ainda tem o nome de \textit{certificado de prata}, indicando que este é realmente um documento 
simplesmente afirmando que o portador deste pedaço de papel é devido um pedaço de prata. Isto é
interessante ver que o texto que indica isso ficou menor com o tempo. O rastreio de \enquote{certificate} 
desapareceu completamente depois de um tempo, sendo substituído pela afirmação tranquilizadora 
de que são notas da reserva federal.

Como mencionado acima, a mesma coisa aconteceu com o ouro. A maior parte do mundo seguia um padrão 
bimetálico, o que significa que as moedas eram feitas principalmente de ouro e prata. Ter certificados 
de ouro, resgatáveis em moedas de ouro, foi sem dúvida uma melhoria tecnológica. O papel é mais 
conveniente, mais leve e, como pode ser dividido arbitrariamente simplesmente imprimindo um número 
menor, é mais fácil dividir em unidades menores.

Para lembrar aos portadores (usuários) que esses certificados eram representativos do ouro 
e da prata reais, eles foram coloridos de acordo e declararam isso claramente no próprio 
certificado. Você pode ler fluentemente a escrita de cima para baixo:

\begin{quotation}\begin{samepage}
\enquote{Isso certifica que foram depositados no tesouro dos Estados Unidos da América cem dólares em moedas 
de ouro pagáveis ao portador sob demanda.}
\end{samepage}\end{quotation}

\begin{figure}
  \centering
  \includegraphics{assets/images/us-gold-cert-100-smaller.png}
  \caption{A 1928 U.S. \$100 gold certificate. Picture cc-by-sa National Numismatic Collection, National Museum of American History.}
  \label{fig:us-gold-cert-100-smaller}
\end{figure}

Em 1963, as palavras \enquote{PAGAMENTO AO PORTADOR DA DEMANDA} foram removidas de todas as notas 
recém-emitidas. Cinco anos depois, o resgate de notas de papel para ouro e prata terminou.

As palavras sugerindo as origens e a idéia por trás do papel-moeda foram removidas. A cor dourada 
desapareceu. Tudo o que restou foi o papel e, com ele, a capacidade do governo de imprimir o máximo 
que quisesse.

Com a abolição do padrão-ouro em 1971, esse truque manual de um século foi concluído. O dinheiro 
se tornou a ilusão que todos compartilhamos até hoje: dinheiro fiduciário. Vale alguma coisa porque 
alguém comandando um exército e operando prisões diz que vale alguma coisa. Como pode ser lido com 
clareza em todas as notas de dólar em circulação hoje, \enquote{ESTA NOTA É LEGAL}. Em outras palavras: 
é valioso porque a nota diz isso.

\begin{figure}
  \centering
  \includegraphics{assets/images/us-dollar-2004.jpg}
  \caption{A 2004 series U.S. twenty dollar note used today. `THIS NOTE IS LEGAL TENDER'}
  \label{fig:us-dollar-2004}
\end{figure}

A propósito, há outra lição interessante nas notas de hoje, oculta à vista de todos. A segunda 
linha diz que este é um curso legal \enquote{PARA TODAS AS DÍVIDAS, PÚBLICAS E PRIVADAS}. O que 
pode ser óbvio para os economistas foi surpreendente para mim: todo dinheiro é dívida. Minha 
cabeça ainda está doendo por causa disso, e deixarei a exploração da relação entre dinheiro e 
dívida como um exercício para o leitor.

\paragraph{}
Como vimos, ouro e prata foram usados como dinheiro por milênios. Com o tempo, moedas feitas 
de ouro e prata foram substituídas por papel. O papel foi lentamente aceito como pagamento. 
Essa aceitação criou uma ilusão - a ilusão de que o papel em si tem valor. O passo final foi 
romper completamente o elo entre a representação e o real: abolir o padrão ouro e convencer a 
todos de que o jornal em si é precioso.

\paragraph{O Bitcoin me ensinou sobre a história do dinheiro e o maior truque da história 
da economia: moeda fiduciária.}

% ---
%
% #### Down the Rabbit Hole
%
% - [Shelling Out: The Origins of Money] by Nick Szabo
% - [Methods of Coin Debasement][coin debasement], [Thaler], [U.S. Silver Certificate][silver certificates], [Bimetallism][bimetallic standard] on Wikipedia
%
% [oldest coin]: https://www.britishmuseum.org/explore/themes/money/the_origins_of_coinage.aspx
% [coin debasement]: https://en.wikipedia.org/wiki/Methods_of_coin_debasement
% [Thaler]: https://en.wikipedia.org/wiki/Thaler
% [Berlin-George]: https://en.wikipedia.org/wiki/File:Bohemia,_Joachimsthaler_1525_Electrotype_Copy._VF._Obverse..jpg
% [silver certificates]: https://en.wikipedia.org/wiki/Silver_certificate_%28United_States%29
% [bimetallic standard]: https://en.wikipedia.org/wiki/Bimetallism
% [Shelling Out: The Origins of Money]: https://nakamotoinstitute.org/shelling-out/
%
% <!-- Wikipedia -->
% [alice]: https://en.wikipedia.org/wiki/Alice%27s_Adventures_in_Wonderland
% [carroll]: https://en.wikipedia.org/wiki/Lewis_Carroll
