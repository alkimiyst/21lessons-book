
\chapter{A verdadeira escassez}
\label{les:2}

\begin{chapquote}{Alice}
\enquote{Já chega - espero não crescer mais\ldots}
\end{chapquote}

iEm geral, o avanço da tecnologia parece tornar as coisas mais abundantes. Mais
e mais pessoas podem desfrutar do que anteriormente eram produtos de luxo.
Em breve, todos viveremos como reis. A maioria de nós já faz. Como Peter Diamandis
escreveu em Abundância~\cite{abundance}: \enquote{A tecnologia é um mecanismo de liberação de recursos. 
Pode tornar o outrora escasso o agora abundante.}

O Bitcoin, uma tecnologia avançada em si, quebra essa tendência e cria
uma nova mercadoria verdadeiramente escassa. Alguns até argumentam que é uma das
as coisas mais escassas do universo. O suprimento não pode ser inflado, 
não importa quanto esforço se gaste para criar mais.


\begin{quotation}\begin{samepage}
\enquote{Apenas duas coisas são genuinamente escassas: tempo e bitcoin.}
\begin{flushright} -- Saifedean Ammous\footnote{Presentation on The Bitcoin Standard~\cite{bitcoinstandard-pres}}
\end{flushright}\end{samepage}\end{quotation}

Paradoxically, it does so by a mechanism of copying. Transactions are
broadcast, blocks are propagated, the distributed ledger is --- well,
you guessed it --- distributed. All of these are just fancy words for
copying. Heck, Bitcoin even copies itself onto as many computers as it
can, by incentivizing individual people to run full nodes and mine new
blocks.

Paradoxalmente, ele o faz por um mecanismo de cópia. As transações são
transmidas, os blocos são propagados, o livro distribuído é --- bem,
você adivinhou --- distribuído. Tudo isso são apenas palavras bonitas para
copia. Caramba, o Bitcoin até se copia em tantos computadores enquanto
incentiva as pessoas a executar nós completos e extrair novos
blocos.

Toda essa duplicação funciona maravilhosamente em um esforço conjunto
para produzir escassez.

\paragraph{Em um tempo de abundância, o Bitcoin me ensinou o que é a verdadeira escassez.}

% ---
%
% #### Through the Looking-Glass
%
% - [Lesson 14: Sound money][lesson14]
%
% #### Down the Rabbit Hole
%
% - [The Bitcoin Standard: The Decentralized Alternative to Central Banking][bitcoin-standard]
% - [Abundance: The Future Is Better Than You Think][Abundance] by Peter Diamandis
% - [Presentation on The Bitcoin Standard][bitcoin-standard-presentation] by Saifedean Ammous
% - [Modeling Bitcoin's Value with Scarcity][planb-scarcity] by PlanB
% - 🎧 [Misir Mahmudov on the Scarcity of Time & Bitcoin][tftc60] TFTC #60 hosted by Marty Bent
% - 🎧 [PlanB – Modelling Bitcoin's digital scarcity through stock-to-flow techniques][slp67] SLP #67 hosted by Stephan Livera
%
% <!-- Through the Looking-Glass -->
% [lesson14]: {{ 'bitcoin/lessons/ch2-14-sound-money' | absolute_url }}
%
% <!-- Down the Rabbit Hole -->
% [Abundance]: https://www.diamandis.com/abundance
% [bitcoin-standard]: http://amzn.to/2L95bJW
% [bitcoin-standard-presentation]: https://www.bayernlb.de/internet/media/de/ir/downloads_1/bayernlb_research/sonderpublikationen_1/bitcoin_munich_may_28.pdf
% [planb-scarcity]: https://medium.com/@100trillionUSD/modeling-bitcoins-value-with-scarcity-91fa0fc03e25
% [tftc60]: https://anchor.fm/tales-from-the-crypt/episodes/Tales-from-the-Crypt-60-Misir-Mahmudov-e3aibh
% [slp67]: https://stephanlivera.com/episode/67
%
% <!-- Wikipedia -->
% [alice]: https://en.wikipedia.org/wiki/Alice%27s_Adventures_in_Wonderland
% [carroll]: https://en.wikipedia.org/wiki/Lewis_Carroll
