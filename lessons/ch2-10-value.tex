\chapter{Valor}
\label{les:10}

\begin{chapquote}{Lewis Carroll, \textit{Alice no País das Maravilhas}}
\enquote{Era o coelho branco, trotando lentamente de volta e olhando ansiosamente
ao longo do caminho, como se tivesse perdido alguma coisa.\ldots}
\end{chapquote}

O valor é um tanto paradoxal e existem várias teorias\footnote{Veja
\textit{Theory of value (economics)} na Wikipedia \cite{wiki:theory-of-value}}
que tentam explicar por que valorizamos certas coisas em detrimento de outras. 
As pessoas estão cientes desse paradoxo há milhares de anos. Como Platão 
escreveu em seu diálogo com Eutidemo, valorizamos algumas coisas porque 
são raras, e não apenas baseadas na necessidade de nossa sobrevivência.

\begin{quotation}\begin{samepage}
\enquote{E se você for prudente, dará o mesmo conselho aos seus alunos
também --- que eles nunca devem conversar com ninguém, exceto você e
um ao outro. Pois é o raro Eutidemo que é precioso, enquanto
a água é mais barata, embora melhor, como disse Pindar.}
\begin{flushright} -- Plato\footnote{Plato, \textit{Euthydemus} \cite{euthydemus}}
\end{flushright}\end{samepage}\end{quotation}

Esse paradoxo de valor\footnote{See \textit{Paradoxo do valor} na Wikipedia
\cite{wiki:paradox-of-value}} mostra algo interessante sobre nós humanos: nós
parecemos valorizar as coisas de forma subjetiva\footnote{See \textit{Teoria subjetiva 
do valor} na Wikipedia \cite{wiki:subjective-theory-of-value}}, mas fazemos isso com 
certos critérios não arbitrários. Algo pode ser \textit{precioso} por várias razões, 
mas as coisas que valorizamos compartilham certas características. Se pudermos copiar 
algo com muita facilidade ou se for naturalmente abundante, não o valorizamos.

Parece que valorizamos algo porque é escasso (ouro, diamantes, tempo), difícil ou 
trabalhoso em produzir, não pode ser substituído (uma foto antiga de um ente querido), 
é útil de uma maneira que permite-nos fazer coisas que de outra forma não poderíamos, 
ou uma combinação delas, como grandes obras de arte.

O Bitcoin é tudo isso: é extremamente raro (21 milhões), cada vez mais difícil de 
produzir (recompensa pela metade), não pode ser substituído (uma chave privada perdida 
é perdida para sempre) e nos permite extrair dele algumas coisas úteis. É sem dúvida a melhor 
ferramenta para transferência de valor através das fronteiras, praticamente resistente à 
censura e confisco no processo, além de ser uma reserva de valor auto-soberana, permitindo 
que indivíduos armazenem sua riqueza independentemente de bancos e governos, apenas para 
citar dois .

\paragraph{O Bitcoin me ensinou que o valor é subjetivo, mas não arbitrário.}

% ---
%
% #### Down the Rabbit Hole
%
% - [Euthydemus] by Plato
% - [Theory of Value][multiple theories], [Paradox of Value][paradox of value], [Subjective Theory of Value][subjective] on Wikipedia
%
% [Euthydemus]: http://www.perseus.tufts.edu/hopper/text?doc=Perseus:text:1999.01.0178:text=Euthyd.
% [Plato]: http://www.perseus.tufts.edu/hopper/text?doc=plat.+euthyd.+304b
%
% <!-- Wikipedia -->
% [multiple theories]: https://en.wikipedia.org/wiki/Theory_of_value_%28economics%29
% [paradox of value]: https://en.wikipedia.org/wiki/Paradox_of_value
% [subjective]: https://en.wikipedia.org/wiki/Subjective_theory_of_value
% [alice]: https://en.wikipedia.org/wiki/Alice%27s_Adventures_in_Wonderland
% [carroll]: https://en.wikipedia.org/wiki/Lewis_Carroll
