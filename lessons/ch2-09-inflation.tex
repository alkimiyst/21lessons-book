\chapter{Inflação}
\label{les:9}

\begin{chapquote}{A Rainha Dos Corações} 
\enquote{Minha querida, aqui devemos correr o mais rápido possível, 
apenas para permanecer no mesmo lugar. E se você deseja ir a qualquer lugar, 
deve correr duas vezes mais rápido que isso.}
\end{chapquote}

Tentar entender a inflação monetária e como um sistema não inflacionário 
como o Bitcoin pode mudar a maneira como fazemos as coisas, foi o ponto 
de partida do minha jornada na economia. Eu sabia que a inflação era a 
taxa em que o dinheiro novo era criado, mas não sabia muito mais além disso.

Enquanto alguns economistas argumentam que a inflação é uma coisa boa, outros 
argumentam que dinheiro \enquote{duro} não pode ser inflado facilmente --- 
como tinhamos nos dias do padrão do ouro --- é essencial para uma economia 
saudável. Bitcoin, com um suprimento fixo de 21 milhões, concorda com este último
argumento.

Geralmente, os efeitos da inflação não são imediatamente óbvios. Dependendo 
da taxa de inflação (assim como de outros fatores), o tempo entre causa e efeito 
pode demorar vários anos. Não apenas isso, mas a inflação afeta diferentes 
grupos de pessoas mais do que outros. Como Henry Hazlitt aponta em 
\textit{Economia em uma lição}:\enquote{A arte da economia consiste em olhar 
não apenas para os efeitos imediatos, mas também para os efeitos mais longos 
de qualquer ato ou política; consiste em rastrear as consequências dessa 
política não apenas para um grupo, mas para todos os grupos.}

Um dos meus momentos pessoais de insight foi a percepção de que emitir nova moeda - 
imprimir mais dinheiro - é uma atividade econômica \textit{completamente} diferente de todas 
as outras atividades econômicas. Enquanto bens e serviços reais produzem valor real 
para pessoas reais, imprimir dinheiro efetivamente faz o oposto: tira valor de 
qualquer pessoa que possua a moeda que está sendo inflada.

\begin{quotation}\begin{samepage}
\enquote{Mera inflação --- isto é, a mera emissão de mais dinheiro, com o
conseqüência de salários e preços mais altos - pode parecer a criação de 
mais demanda. Mas em termos da produção e troca reais de coisas reais, não é.}
\begin{flushright} -- Henry Hazlitt\footnote{Henry Hazlitt, \textit{Economics in One Lesson} \cite{hazlitt}}
\end{flushright}\end{samepage}\end{quotation}


A força destrutiva da inflação se torna óbvia assim que um pouco de inflação se 
transforma em \textit{muito}. Se o dinheiro hiperinflar as coisas ficam feias bem  
rápido.\footnote{\url{https://en.wikipedia.org/wiki/Hyperinflation}
\cite{wiki:hyperinflation}} À medida que a moeda inflável se desfaz, 
ela falha em armazenar valor ao longo do tempo e as pessoas se apressam em pôr 
as mãos em qualquer mercadoria que possa funcionar.

\paragraph{}
Outra consequência da hiperinflação é que todo o dinheiro que as pessoas economizaram 
ao longo de sua vida desaparecerá efetivamente. O papel-moeda na sua carteira ainda 
estará lá, é claro. Mas vai ser exatamente isso: um papel sem valor.

\begin{figure}
  \includegraphics{assets/images/children-playing-with-money.png}
  \caption{Hyperinflation in the Weimar Republic (1921-1923)}
  \label{fig:children-playing-with-money}
\end{figure}

\paragraph{}
O dinheiro também diminui de valor com a chamada inflação \enquote{leve}. 
Isso acontece devagar o suficiente para que a maioria das pessoas não 
perceba a diminuição de seu poder de compra. E uma vez que as impressoras 
estejam funcionando, a moeda pode ser inflada facilmente, e o que costumava 
ser uma inflação leve pode se transformar em um forte copo de inflação 
pressionando um botão. 
Como Friedrich Hayek apontou em um de seus ensaios, a inflação leve geralmente 
leva à inflação total.

\begin{quotation}\begin{samepage}
\enquote{`Uma inflação leve e estável não pode ajudar - ela pode levar apenas a uma inflação definitiva.}
\begin{flushright} -- Friedrich Hayek\footnote{Friedrich Hayek, \textit{1980s
Unemployment and the Unions} \cite{hayek-inflation}}
\end{flushright}\end{samepage}\end{quotation}

A inflação é particularmente desonesta, pois favorece aqueles que estão mais próximos
às impressoras. Leva tempo para o dinheiro recém-criado
circular e preços para ajustar, por isso, se você conseguir pôr as mãos em 
mais dinheiro antes de todo mundo desvalorizar você estará à frente da
curva inflacionária. É também por isso que a inflação pode ser vista como um
imposto porque, no final, os governos lucram com ele, enquanto todo mundo acaba pagando o preço.

\begin{quotation}\begin{samepage}
\enquote{Não creio que seja exagero dizer que a história é em grande parte uma história da 
inflação e, geralmente, de inflação projetada pelos governos para o ganho de governos.}
\begin{flushright} -- Friedrich Hayek\footnote{Friedrich Hayek, \textit{Good Money} \cite{hayek-good-money}}
\end{flushright}\end{samepage}\end{quotation}

\newpage

Até agora, todas as moedas controladas pelo governo acabaram sendo
substituídas ou desmoronou completamente. Por menor que seja a taxa de
inflação, o crescimento \enquote{fixo} é apenas outra maneira de dizer crescimento exponencial. 
Na natureza e na economia, todos os sistemas que crescem exponencialmente acabarão 
por se estabilizar ou sofrer um colapso catastrófico.

\paragraph{}
\enquote{Isso não pode acontecer no meu país,} é o que você provavelmente está pensando. 
Você não acha que se você é da Venezuela, que atualmente sofre de hiperinflação. Com uma taxa
de inflação acima de 1 milhão por cento, o dinheiro é basicamente inútil.. \cite{wiki:venezuela}

\paragraph{}
Isso pode não acontecer nos próximos dois anos ou na moeda específica
usado no seu país. Mas uma olhada na lista de históricos de moedas\footnote{See 
\textit{List of historical currencies} on Wikipedia.
\cite{wiki:historical-currencies}} mostra que isso inevitavelmente acontecerá 
ao longo de um período de tempo suficiente. Lembro-me e usei muitos dos listados: o
Xelim austríaco, o marco alemã, a lira italiana, o franco francês, a
Libra irlandesa, o dinar croata, etc. Minha avó até usou o krone austro-húngaro
. À medida que o tempo passa, as moedas atualmente em uso\footnote{See
\textit{List of currencies} on Wikipedia \cite{wiki:list-of-currencies}} vai
lenta mas seguramente, vá para seus respectivos cemitérios. Eles vão hiperinflar ou
ser substituído. Em breve serão moedas históricas. Vamos torna-los obsoletos.

\begin{quotation}\begin{samepage}
\enquote{A história mostrou que os governos sucumbirão inevitavelmente ao
tentação de aumentar o suprimento de dinheiro.}
\begin{flushright} -- Saifedean Ammous\footnote{Saifedean Ammous, \textit{The Bitcoin
Standard} \cite{bitcoin-standard}}
\end{flushright}\end{samepage}\end{quotation}

\newpage

Porque o Bitcoin é diferente ? Em contraste com as moedas impostas pelo governo,
bens monetários que não são regulados pelos governos, mas pelas leis da física
\footnote{Gigi, \textit{Bitcoin's Energy Consumption - A shift in
perspective} \cite{gigi:energy}}, tendem a sobreviver e até manter seus respectivos
valor ao longo do tempo. O melhor exemplo disso até agora é o ouro, que, como o
apropriadamente chamado \textit{Proporção Ouro / Processo Decente}\footnote{A história mostra que o
preço de uma onça de ouro é igual ao preço de um terno masculino decente, de acordo com os
gerentes de investimento da Sionna\cite{web:gold-to-decent-suite-ratio}} mostra, que está 
mantendo seu valor por centenas e até milhares de anos. Talvez não seja perfeitamente 
\enquote{estável} --- um conceito questionável em primeiro lugar ---  mas o valor que ele tem 
será pelo menos na mesma ordem de magnitude.

Se um bem ou moeda mantiver seu valor bem ao longo do tempo e do espaço, será considerado \textit{forte}. 
Se não puder manter seu valor, porque se deteriora ou infla facilmente, é considerada uma moeda \textit{fraca}. 
O conceito de dureza é essencial para entender o Bitcoin e merece um exame mais completo. Voltaremos a ele 
na última lição econômica: dinheiro vivo, são e forte.

\paragraph{}
À medida que mais e mais países sofrem de hiperinflação, mais e mais pessoas terão que enfrentar a 
realidade do dinheiro forte e do dinheiro fraco. Se tivermos sorte, talvez até alguns banqueiros 
centrais sejam forçados a reavaliar suas políticas monetárias. Aconteça o que acontecer, as idéias 
que obtive graças ao Bitcoin provavelmente serão inestimáveis, independentemente do resultado.

\paragraph{O Bitcoin me ensinou sobre o imposto oculto da inflação e a catástrofe da hiperinflação.}

% ---
%
% #### Down the Rabbit Hole
%
% - [Economics in One Lesson][Henry Hazlitt] by Henry Hazlitt
% - [1980's Unemployment and the Unions][unions] by Friedrich Hayek
% - [Good Money, Part II][good-money]: Volume Six of the Collected Works of F.A. Hayek
% - [The Bitcoin Standard] by Saifedean Ammous
% - [Hyperinflation][hyperinflates], [economic crisis in Venezuela][wiki-venezuela], [list of historical currencies], [list of currencies][currently in use] on Wikipedia
%
% [unions]: https://books.google.com/books/about/1980s_unemployment_and_the_unions.html?id=xM9CAQAAIAAJ
% [good-money]: https://books.google.com/books?id=l_A1vVIaYBYC
%
% [Henry Hazlitt]: https://mises.org/library/economics-one-lesson
% [hyperinflates]: https://en.wikipedia.org/wiki/Hyperinflation
% [inflation cannot help]: https://books.google.com/books?id=zZu3AAAAIAAJ&dq=%22only+while+it+accelerates%22&focus=searchwithinvolume&q=%22steady+inflation+cannot+help%22
% [history of inflation]: https://books.google.com/books?id=l_A1vVIaYBYC&pg=PA142&dq=%22history+is+largely+a+history+of+inflation%22&hl=en&sa=X&ved=0ahUKEwi90NDLrdnfAhUprVkKHUx1CmIQ6AEIKjAA#v=onepage&q=%22history%20is%20largely%20a%20history%20of%20inflation%22&f=false
% [wiki-venezuela]: https://en.wikipedia.org/wiki/Crisis_in_Venezuela#Economic_crisis
% [by the laws of physics]: https://link.medium.com/9fzq2L0J3S
% [\textit{Gold-to-Decent-Suit Ratio}]: https://www.businesswire.com/news/home/20110819005774/en/History-Shows-Price-Ounce-Gold-Equals-Price
% [The Bitcoin Standard]: https://thesaifhouse.wordpress.com/book/
%
% <!-- Wikipedia -->
% [alice]: https://en.wikipedia.org/wiki/Alice%27s_Adventures_in_Wonderland
% [carroll]: https://en.wikipedia.org/wiki/Lewis_Carroll
