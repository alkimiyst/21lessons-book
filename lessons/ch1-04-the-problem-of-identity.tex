\chapter{O Problema da Identidade}
\label{les:4}

\begin{chapquote}{Lewis Carroll, \textit{Alice no País das Maravilhas}}
  \enquote{Como vai você?} disse a lagarta.
\end{chapquote}

Nic Carter, em homenagem ao tratamento de Thomas Nagel da mesma
pergunta em relação a um morcego, escreveu uma excelente peça que discute
a seguinte pergunta: Como é ser um bitcoin? Ele
mostra brilhantemente que blockchains públicos abertos em geral e Bitcoin
sofrem, em particular, do mesmo dilema que o navio de Teseu\footnote{na
metafísica da identidade, a nave de Teseu é um experimento mental que
levanta a questão de saber se um objeto que teve todos os seus componentes
substituído permanece fundamentalmente o mesmo objeto.~\cite{wiki:theseus}}: qual
Bitcoin é o verdadeiro Bitcoin?

\begin{quotation}\begin{samepage}
\enquote{Considere o quão pouca persistência os componentes do Bitcoin têm. o
código foi todo reformulado, alterado e expandido ao ponto de que
mal se assemelha à sua versão original. [...] o registro de quem
possui o que, o próprio livro, é praticamente a única característica persistente
da rede [...]
Para ser considerado verdadeiramente sem líder, você deve renunciar à 
solução fácil de ter uma entidade que possa designar uma cadeia de blocos como a
legítima.}
\begin{flushright} -- Nic Carter\footnote{Nic Carter, \textit{What is it like to be a bitcoin?} \cite{bitcoin-identity}}
\end{flushright}\end{samepage}\end{quotation}

Parece que o avanço da tecnologia continua nos forçando a tomar
essas questões filosóficas a sério. Mais cedo ou mais tarde, dirigir sozinho
os carros serão confrontados com versões reais do dilema do bonde,
forçando-os a tomar decisões éticas sobre cujas vidas são importantes e
de quem não.

Criptomoedas, especialmente desde o primeiro hard-fork contencioso,
nos força a pensar e concordar com a metafísica da identidade.
Curiosamente, os dois maiores exemplos que temos até agora levaram a duas
respostas diferentes. Em 1 de agosto de 2017, o Bitcoin se dividiu em dois campos. o
O mercado decidiu que a cadeia de blocos inalterada é o Bitcoin original. Um ano
antes, em 25 de outubro de 2016, o Ethereum se dividiu em dois campos.

Se adequadamente descentralizadas, as questões colocadas pelo \textit{Navio de Teseu}
terão de ser respondidas perpetuamente enquanto essas redes de
transferência de valor existirem.

\paragraph{O Bitcoin me ensinou que a descentralização contradiz a identidade.}

% ---
%
% #### Down the Rabbit Hole
%
% - [What Is It Like to be a Bat?][in regards to a bat] by Thomas Nagel
% - [What is it like to be a bitcoin?] by Nic Carter
% - [Ship of Theseus], [trolley problem] on Wikipedia
%
% [in regards to a bat]: https://en.wikipedia.org/wiki/What_Is_it_Like_to_Be_a_Bat%3F
% [What is it like to be a bitcoin?]: https://medium.com/s/story/what-is-it-like-to-be-a-bitcoin-56109f3e6753
% [Ship of Theseus]: https://en.wikipedia.org/wiki/Ship_of_Theseus
% [trolley problem]: https://en.wikipedia.org/wiki/Trolley_problem
%
% <!-- Wikipedia -->
% [alice]: https://en.wikipedia.org/wiki/Alice%27s_Adventures_in_Wonderland
% [carroll]: https://en.wikipedia.org/wiki/Lewis_Carroll
