\chapter{Replicação e Localidade}
\label{les:3}

\begin{chapquote}{Lewis Carroll, \textit{Alice no País das Maravilhas}}
Depois veio uma voz furiosa -- a do coelho -- \enquote{Pat, Pat! cade você?}
\end{chapquote}

Deixando de lado a mecânica quântica, a localidade não é um problema no mundo físico.
Essa questão \textit{\enquote{Onde está X?}} pode ser respondida de maneira significativa, não
importa se X é uma pessoa ou um objeto. No mundo digital, a questão de
 \textit{onde} já é mais complicada, mas não impossível de responder. Onde estão seus emails,
, sério? Uma péssima resposta seria \enquote{na nuvem},  que é
apenas o computador de alguem. Ainda assim, se você quiser rastrear todos os
discos de armazenamento com seus e-mails, você poderia, em teoria, localizar
eles.

Com o bitcoin, a questão de \enquote{onde} é \textit{realmente} complicada. Onde,
exatamente, estão seus bitcoins?

\begin{quotation}\begin{samepage}
\enquote{Abri os olhos, olhei em volta e perguntei o inevitável, o
tradicional, a pergunta pós-operatória lamentavelmente original: `Onde
eu estou?'}
\begin{flushright} -- Daniel Dennett\footnote{Daniel Dennett, \textit{Where Am I?}~\cite{where-am-i}}
\end{flushright}\end{samepage}\end{quotation}

O problema é duplo: primeiro, o livro de contabilidade distribuído é distribuído por
replicação completa, o que significa que os registros estão em todos os lugares. Segundo, não há
bitcoins. Não apenas fisicamente, mas \textit{tecnicamente}.

O Bitcoin mantém o controle de um conjunto de resultados de transações não gastas, sem
tendo que se referir a uma entidade que representa um bitcoin.  a existência de um bitcoin 
é inferida olhando-se o conjunto de saídas de transação e chamando cada entrada com 100 milhões de base
unidades de um bitcoin.

\begin{quotation}\begin{samepage}
\enquote{Onde está, neste momento, em trânsito? [...] primeiro, não tem
bitcoins. Simplesmente não há. Eles não existem. Existem livros
entradas em um livro compartilhado [...] Eles não existem em nenhuma
localização física. O livro existe em todos os locais físicos,
essencialmente. Geografia não faz sentido aqui --- não vai
ajudá-lo a descobrir sua política aqui.}
\begin{flushright} -- Peter Van Valkenburgh\footnote{Peter Van Valkenburgh on the \textit{What Bitcoin Did} podcast, episode 49 \cite{wbd049}}
\end{flushright}\end{samepage}\end{quotation}

Então, o que você realmente possui quando diz \textit{\enquote{Eu tenho um bitcoin}} se
não existem bitcoins? Bem, lembre-se de todas essas palavras estranhas que você estava
forçado a escrever pela carteira que você usou? Acontece que essas palavras mágicas são
o que você possui: um feitiço mágico! \footnote{The Magic Dust of Cryptography: How digital
information is changing our society \cite{gigi:magic-spell}} que pode ser usado
para adicionar algumas entradas no livro de contabilidade pública --- as chaves para
 \enquote{mover} alguns bitcoins.
É por isso que, para todos os efeitos, suas chaves privadas \textit{são} seus 
bitcoins. Se você acha que estou inventando tudo isso, sinta-se à vontade para me enviar sua
chaves privadas.

\paragraph{Bitcoin taught me that locality is a tricky business.}

% ---
%
% #### Through the Looking-Glass
%
% - [The Magic Dust of Cryptography: How digital information is changing our society][a magic spell]
%
% #### Down the Rabbit Hole
%
% - [Where Am I?][Daniel Dennett] by Daniel Dennett
% - 🎧 [Peter Van Valkenburg on Preserving the Freedom to Innovate with Public Blockchains][wbd049] WBD #49 hosted by Peter McCormack
%
% <!-- Through the Looking-Glass -->
% [a magic spell]: 
%
% <!-- Down the Rabbit Hole -->
% [Daniel Dennett]: https://www.lehigh.edu/~mhb0/Dennett-WhereAmI.pdf
% [1st Amendment]: https://en.wikipedia.org/wiki/First_Amendment_to_the_United_States_Constitution
% [wbd049]: https://www.whatbitcoindid.com/podcast/coin-centers-peter-van-valkenburg-on-preserving-the-freedom-to-innovate-with-public-blockchains
%
% <!-- Wikipedia -->
% [alice]: https://en.wikipedia.org/wiki/Alice%27s_Adventures_in_Wonderland
% [carroll]: https://en.wikipedia.org/wiki/Lewis_Carroll
