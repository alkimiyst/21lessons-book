\chapter{O Poder da Liberdade de Expressão}
\label{les:6}

\begin{chapquote}{Lewis Carroll, \textit{Alice no País das Maravilhas}}
\enquote{Perdão?} falou o rato, franzindo a testa, mas muito educadamente, \enquote{Você fala?}
\end{chapquote}

Bitcoin é uma ideia. Uma ideia que, na sua forma atual, é a
manifestação de um maquinario puramente alimentado por texto. Todos os aspectos do
Bitcoin é texto: o whitepaper é texto. O software que é executado por
seus nós são texto. O livro contábil é texto. Transações são texto. Público e
chaves privadas são texto. Cada aspecto do Bitcoin é texto e, portanto,
equivalente a fala.

\begin{quotation}\begin{samepage}
\enquote{O Congresso não fará nenhuma lei que respeite o estabelecimento de religião,
ou proibindo o livre exercício do mesmo; ou abreviando a liberdade de
discurso ou da imprensa; ou o direito das pessoas pacificamente se
reunirem e solicitarem ao Governo uma reparação de queixas.}
\begin{flushright} -- Primeira Emenda à Constituição dos EUA
\end{flushright}\end{samepage}\end{quotation}

Embora a batalha final das guerras criptográficas\footnote{As \textit{Guerras Criptográficas}
é um nome não oficiali para as tentativas dos EUA e os governos aliados minar o uso e evolução de 
criptografia.~\cite{eff-cryptowars}~\cite{wiki:cryptowars}} não tenha sido travada
ainda, será muito difícil criminalizar uma idéia, muito menos uma idéia que
baseia-se na troca de mensagens de texto. Toda vez que um governo tenta
proibir texto ou discurso, seguimos um caminho de absurdo que inevitavelmente leva
a abominações como números ilegais\footnote{Um número ilegal é um número que
representa informações que são ilegais possuir, pronunciar, propagar ou
caso contrário, transmitir em alguma jurisdição legal.\cite{wiki:illegal-number}} e números primos ilegais
\footnote{Um número primo ilegal é um número primo que representa
informações cuja posse ou distribuição seja proibida em algumas
jurisdições. Um dos primeiros números primos ilegais foi encontrado em 2001. Quando
interpretado de uma maneira particular, descreve um programa de computador que ignora
o esquema de gerenciamento de direitos digitais usado em DVDs. A distribuição de tal
programa nos Estados Unidos é ilegal sob o Digital Millennium Copyright
Act. Um número primo ilegal é um tipo de número ilegal.\cite{wiki:illegal-prime}}.

Enquanto houver uma parte do mundo em que a fala seja livre como em
\textit{liberdade}, Bitcoin é imparável.

\begin{quotation}\begin{samepage}
\enquote{Não faz sentido em nenhuma transação Bitcoin que o Bitcoin deixe de ser
\textit{texto}. Isso é \textit{texto}, o tempo todo. [...] Bitcoin é	
\textit{texto}. Bitcoin é \textit{fala}. Ele não pode ser regulado em um país
livre como os EUA que tem garantias inalienaveis pelos direitos e pela Primeira Emenda
que explicitamente exclui a supervião governamental sobre as publicações.}
\begin{flushright} -- Beautyon\footnote{Beautyon, \textit{Why America can't regulate
Bitcoin} \cite{america-regulate-bitcoin}}
\end{flushright}\end{samepage}\end{quotation}

\paragraph{O Bitcoin me ensinou que, em uma sociedade livre, liberdade de expressão e software livre
são imparáveis.}

% ---
%
% #### Through the Looking-Glass
%
% - [The Magic Dust of Cryptography: How digital information is changing our society][a magic spell]
%
% #### Down the Rabbit Hole
%
% - [Why America can't regulate Bitcoin][Beautyon] by Beautyon
% - [First Amendment to the United States Constitution][1st Amendment], [Crypto Wars], [illegal numbers], [illegal primes] on Wikipedia
%
% <!-- Through the Looking-Glass -->
% [a magic spell]: 
%
% <!-- Down the Rabbit Hole -->
% [1st Amendment]: https://en.wikipedia.org/wiki/First_Amendment_to_the_United_States_Constitution
% [Crypto Wars]: https://en.wikipedia.org/wiki/Crypto_Wars
% [illegal numbers]: https://en.wikipedia.org/wiki/Illegal_number
% [illegal primes]: https://en.wikipedia.org/wiki/Illegal_prime
% [Beautyon]: https://hackernoon.com/why-america-cant-regulate-bitcoin-8c77cee8d794
%
% <!-- Wikipedia -->
% [alice]: https://en.wikipedia.org/wiki/Alice%27s_Adventures_in_Wonderland
% [carroll]: https://en.wikipedia.org/wiki/Lewis_Carroll
