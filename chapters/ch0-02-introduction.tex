\chapter*{Introdução}
\label{ch:introduction}

\begin{chapquote}{Lewis Carroll, \textit{Alice no País das Maravilhas}}
\enquote{Mas eu não quero estar entre as pessoas loucas,} Alice comentou. \enquote{Oh, você não pode evitar isso,} disse o Gato: \enquote{estamos todos loucos aqui. Eu estou louco. Você é louca.} \enquote{Como
você sabe que eu sou louca?} disse Alice. \enquote{Você deve ser,} disse o Gato, \enquote{ou você nunca teria vindo aqui.}
\end{chapquote}

Em outubro de 2018, Arjun Balaji fez a pergunta inócua,
\textit{O que vocé aprendeu do Bitcoin?} Depois de tentar responder a essa
pergunta em um pequeno tweet, e falhado miseravelmente, percebi que as coisas que
aprendi são numerosas demais para responder rapidamente, se é que o fazem.

As coisas que eu aprendi são, claro, sobre Bitcoin - ou ao menos relacionadas a isso. De qualquer forma, enquanto alguns aspectos internos do Bitcoin são explicados, as lições a seguir não são uma explicação de como o Bitcoin funciona ou o que é,
eles podem, no entanto, ajudar a explorar algumas das coisas que o Bitcoin toca:
questões filosóficas, realidades econômicas e inovações tecnológicas.

\begin{center}
  \includegraphics[width=7cm]{assets/images/the-tweet.png}
\end{center}

As \textit{21 Lições} estão estruturadas em pacotes de sete, resultando em três
capítulos. Cada capítulo analisa o Bitcoin através de uma lente diferente, extraindo
as lições que podem ser aprendidas ao inspecionar essa estranha rede de um ângulo diferente.

\paragraph{\hyperref[ch:philosophy]{Capítulo 1}}{explora os ensinamentos filosóficos do Bitcoin. A interação de imutabilidade e mudança, o conceito de verdadeira escassez, a imaculada concepção do Bitcoin, o problema de identidade, a contradição da replicação e localidade, o poder da liberdade de expressão e os limites do conhecimento.
}

\paragraph{\hyperref[ch:economics]{Capítulo 2}}{explora os ensinamentos econômicos
do Bitcoin. Lições sobre ignorância financeira, inflação, valor, dinheiro e a
história do dinheiro, banco de reservas fracionárias e como o Bitcoin está reintroduzindo
"dinheiro vivo" de uma maneira astuta e indireta.}

\paragraph{\hyperref[ch:technology]{Capítulo 3}}{explora algumas das lições
aprendidas examinando a tecnologia do Bitcoin. Por que há força em
números, reflexões sobre confiança, por que dizer tempo dá trabalho, como andar devagar
e não quebrar coisas é um recurso e não um bug, o que a criação do Bitcoin pode
contar-nos sobre privacidade, por que cypherpunks escrevem código (e não leis) e quais metáforas podem ser úteis para explorar o futuro do Bitcoin.}

~

Cada lição contém várias citações e links ao longo do texto. Se uma ideia vale a pena explorar com mais detalhes, você pode seguir os links para trabalhos relacionados nas
notas de rodapé ou na bibliografia.

Embora algum conhecimento prévio sobre Bitcoin seja benéfico, espero que essas
lições possam ser digeridas por qualquer leitor curioso. Enquanto algumas se relacionam,
cada lição deve ser independente e pode ser lida independentemente. Eu
fiz o possível para evitar o jargão técnico, mesmo que em alguns domínios específicos
o vocabulário é inevitável.

Espero que meus escritos sirvam de inspiração para que outras pessoas procurem
surgir e examinar algumas das questões mais profundas que o Bitcoin levanta. Minha própria inspiração veio de uma infinidade de autores e criadores de conteúdo para todos
Sou eternamente grato.

Por último, mas não menos importante: meu objetivo ao escrever isso não é convencê-lo de nada.
Meu objetivo é fazer você pensar e mostrar que há muito mais no Bitcoin
do que aparenta. Eu não posso nem dizer o que é o Bitcoin ou o que o Bitcoin vai
te ensinar. Você terá que descobrir isso por si mesmo.

\begin{quotation}\begin{samepage}
\enquote{Depois disso, não há como voltar atrás. Você toma a pílula azul - a
a história termina, você acorda em sua cama e acredita no que quer
acreditar. Você toma a pílula vermelha\footnote{a pílula \textit{laranja}} --- você fica no país das maravilhas e eu te mostro quão fundo o buraco do coelho vai.}
\begin{flushright} -- Morpheus
\end{flushright}\end{samepage}\end{quotation}

\begin{figure}
  \includegraphics{assets/images/bitcoin-orange-pill.jpg}
  \caption*{Lembre-se: tudo o que estou oferecendo é a verdade. Nada mais.}
  \label{fig:bitcoin-orange-pill}
\end{figure}

%
% [Morpheus]: https://en.wikipedia.org/wiki/Red_pill_and_blue_pill#The_Matrix_(1999)
% [this question]: https://twitter.com/arjunblj/status/1050073234719293440
%
% <!-- Internal -->
% [chapter1]: {{ 'bitcoin/lessons/ch1-00-philosophy' | absolute_url }}
% [chapter2]: {{ 'bitcoin/lessons/ch2-00-economics' | absolute_url }}
% [chapter3]: {{ 'bitcoin/lessons/ch3-00-technology' | absolute_url }}
%
% <!-- Wikipedia -->
% [alice]: https://en.wikipedia.org/wiki/Alice%27s_Adventures_in_Wonderland
% [carroll]: https://en.wikipedia.org/wiki/Lewis_Carroll
